\newenvironment{abstracts} {\begin{alwayssingle} \pagestyle{empty}
  \begin{center}
  \vspace*{1.5cm}
  {\Large \bfseries  Abstract}
  \end{center}
  \vspace{0.5cm} \begin{quote}}
{\end{quote}\end{alwayssingle}}
\begin{abstracts}

 Advancements in machine learning have significantly enhanced the capabilities of language-based applications. However, traditional methods of instructing and integrating language models with hand-crafted prompts within complex applications can become tedious and lack robust self-improvement capabilities.  This thesis investigates the potential of DSPy, a novel framework designed to compile declarative language model calls into continuously improving pipelines. We focus on the realm of conversational AI agents, conducting a case study to illuminate how DSPy can streamline development. The experimental analysis evaluates how DSPy's unique self-improvement paradigm impacts user experience, model personalization using application-specific datasets, and resource optimization within conversational AI contexts.\\ \\
\textbf{Keywords : } {DSPy, self-improvement, conversational AI, personalization, language models, machine learning, optimization}
\end{abstracts}