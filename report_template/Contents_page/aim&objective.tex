\chapter{Aim and Objective}
\section{Aim}\\
\\
\textbf{Evaluate DSPy's potential:} Test whether DSPy is a viable framework to enhance the development process of conversational AI agents in particular. The focus is not just building an exemplary AI agent, but understanding how DSPy's unique traits impact the process.\\
\\
\textbf{Emphasize Data-Driven Development:} Assess if building an agent in DSPy aligns with a data-driven approach. Can user data directly drive changes that improve the application?\\
\\
\textbf{Focus on Self-Improvement:} This is the core value proposition of DSPy; the aim is to analyze whether an AI built using it can actively get better based on its own experience.\\
\\
\textbf{Aim for User-Centricity:} Ultimately, the goal is to understand if all of this results in a better user experience when interacting with these AI agents.
\section{Objectives}\\
\\
The objectives break down this aim into actionable and measurable steps:
\begin{itemize}
    \item Implement a prototype conversational AI agent: This serves two purposes:
    \begin{itemize}
        \item Proving DSPy can handle a real-world conversational AI task.
        \item Evaluating DSPy on different publicly available datasets.
    \end{itemize}
    \item Develop and integrate self-improvement algorithms: This is the technical heart of the research. Implementing methods by which the model/pipeline learns from the collected data to improve itself.
    \item Conduct experiments to measure the impact of DSPy's pipeline improvements:
    \begin{itemize}
        \item Beyond anecdotal success, it's essential to run experiments that isolate DSPy's improvements and quantify their positive effect on the user experience, model performance, and resource efficiency. 
    \end{itemize}
\end{itemize}
